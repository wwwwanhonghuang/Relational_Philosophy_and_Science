%%%%%%%%%%%%%%%%%%%%%%preface.tex%%%%%%%%%%%%%%%%%%%%%%%%%%%%%%%%%%%%%%%%%
% sample preface
%
% Use this file as a template for your own input.
%
%%%%%%%%%%%%%%%%%%%%%%%% Springer Nature %%%%%%%%%%%%%%%%%%%%%%%%

\preface


The story of the modern mind is one of a great displacement. For ages, a supreme God occupied the central, sovereign position in the human spirit, providing a foundation for meaning, morality, and order. The Enlightenment, that brave and brilliant turn, dethroned this divine sovereign. In its place, we enthroned a new regent: Reason.

We invested our faith in the symbols of natural science, believing their precise language could lead us to all worthwhile knowledge and mastery over the world. Concurrently, we trusted that the production and reproduction of social symbols—through culture, ideology, and the arts—would be sufficient to populate and sustain our spiritual homeland. We built intricate systems of thought, expecting them to bear the weight of our deepest human needs for purpose, connection, and transcendence.

Today, the edifice of this rationalist faith is showing catastrophic strain. The symbols of natural science, for all their power to explain and manipulate the physical world, fall silent before questions of ultimate value and meaning. The symbols produced by our social systems, increasingly commodified and fragmented, have proven to be a thin gruel for the human soul. The rational structures we erected, both scientific and social, are proving inadequate to support the immense and complex architecture of the human spirit. The result is a crisis of nihilism, a pervasive emptiness at the heart of our collective home.

It is from the heart of this crisis that the present book emerges. Its central motivation is the conviction that to confront this void, we cannot simply continue on the same path. A fundamental intervention is required, not upon the world as an external object, but upon the very structures of our knowing and being. We must consciously and deliberately steer our development toward a new mode of existence: a structural existence, or relational existence, where our individual and collective identities are understood and lived as dynamic nodes within networks of relation, rather than as isolated atoms. This shift must manifest not only in our personal lives but also in the foundational practices of our sciences, reshaping both our cognition and our praxis.

Therefore, this work is more than a treatise; it is a blueprint for a philosophical and scientific renovation of our world.

In a commitment to the very principles it advocates, this book is offered as an open-source project. 
Its writing has not been a solitary endeavor but a concrete practice of relational existence—a network of ideas, critiques, 
and inspirations woven together through dialogue. It is our hope that in this same spirit, it will be read, challenged, expanded, 
and embodied by a community of thinkers and practitioners, becoming a living artifact of our collective 
movement toward a more connected, meaningful, and resilient future.

 

\vspace{\baselineskip}
\begin{flushright}\noindent
Place(s),\hfill {\it Firstname  Surname}\\
month year\hfill {\it Firstname  Surname}\\
\end{flushright}


